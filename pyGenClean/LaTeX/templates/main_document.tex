\documentclass[10pt,twoside,english]{scrartcl}

\usepackage[T1]{fontenc}
\usepackage[utf8]{inputenc}

\usepackage[letterpaper]{geometry}
\geometry{verbose,tmargin=2cm,bmargin=2cm,lmargin=2cm,rmargin=2cm,footskip=1cm}

\usepackage{fancyhdr}
\pagestyle{fancy}

% Ordinal numbers
\usepackage{engord}

\setlength{\parindent}{0cm}

\usepackage{color}

\usepackage{babel}

\usepackage[unicode=true,
 bookmarks=true,bookmarksnumbered=false,bookmarksopen=false,
 breaklinks=false,pdfborder={0 0 0},backref=false,colorlinks=true]
 {hyperref}
\hypersetup{pdftitle={\VAR{ project_name }},
 pdfauthor={\VAR{ report_author }},
 linkcolor={link_blue},citecolor={link_blue},urlcolor={link_blue}}

\makeatletter
%%%%%%%%%%%%%%%%%%%%%%%%%%%%%% User specified LaTeX commands.
\usepackage[activate={true,nocompatibility},final,tracking=true,kerning=true,spacing=true,factor=1100,stretch=10,shrink=10]{microtype}
% activate={true,nocompatibility} - activate protrusion and expansion
% final - enable microtype; use "draft" to disable
% tracking=true, kerning=true, spacing=true - activate these techniques
% factor=1100 - add 10% to the protrusion amount (default is 1000)
% stretch=10, shrink=10 - reduce stretchability/shrinkability (default is 20/20)

% Some color definitions
\usepackage{xcolor}
\definecolor{blue}{HTML}{0099CC}
\definecolor{red}{HTML}{CC0000}
\definecolor{green}{HTML}{669900}
\definecolor{link_blue}{HTML}{21759B}

% Caption and subcaptions
\usepackage[bf]{caption}
\usepackage{subcaption}

% We want images
\usepackage{float}
\usepackage{graphicx}

% Enables page in landscape mode
\usepackage{pdflscape}

% Color in tables + dashed line
\usepackage{longtable}
\usepackage{colortbl}
\usepackage{arydshln}

% Uses urls
\usepackage{url}

% The Report number...
\newcommand{\ProjectName}{\VAR{ project_name }}

% Fancy header
\pagestyle{fancy}
\fancyhf{}
\fancyfoot[RO,LE]{{\footnotesize \texttt{pyGenClean} version \VAR{ pygenclean_version }}}
\fancyfoot[LO,RE]{{\footnotesize\ProjectName}}
\fancyfoot[CO,CE]{{\footnotesize\thepage}}
\renewcommand{\headrulewidth}{0pt}
\renewcommand{\footrulewidth}{1pt}

% The plain fancy header
\fancypagestyle{plain}{
    \fancyhf{} % clear all header and footer fields
    \renewcommand{\headrulewidth}{0pt}
    \renewcommand{\footrulewidth}{0pt}
}

% Table's numerotations are in Roman numbers
\renewcommand{\thetable}{\Roman{table}}

% Subfigure numbering
\renewcommand{\thesubfigure}{\Alph{subfigure}}

\let\tempone\itemize
\let\temptwo\enditemize
\renewenvironment{itemize}{\tempone\setlength{\itemsep}{0pt}}{\temptwo}

\let\ttempone\enumerate
\let\ttemptwo\endenumerate
\renewenvironment{enumerate}{\ttempone\setlength{\itemsep}{0pt}}{\ttemptwo}

\let\tttempone\description
\let\tttemptwo\enddescription
\renewenvironment{description}{\tttempone\setlength{\itemsep}{0pt}}{\tttemptwo}

\makeatother

\begin{document}

% The title page
\title{Genetic Data Clean Up}
\subtitle{\ProjectName}
\author{\VAR{ report_author }}
\date{\VAR{ month } \engordnumber{\VAR{ day }}, \VAR{ year }}
\maketitle

% The table of contents
\microtypesetup{protrusion=false} % disables protrusion locally in the document
\tableofcontents{}
\listoffigures{}
\microtypesetup{protrusion=true} % enables protrusion
\clearpage
\newpage

\section{Background}
\VAR{ background_content }


\section{Methods}
Plink was used for the data cleanup procedure. The automated script
\texttt{pyGenClean} version \VAR{ pygenclean_version }~\cite{pyGenClean} was
used to launch the analysis. The following files were used as input.

\begin{itemize}
\BLOCK{ for fn in initial_files }
\item \path{\VAR{ fn }}
\BLOCK{ endfor }
\end{itemize}

Briefly, the clean up procedure was as follow:
\begin{enumerate}

\input{\VAR{ steps_filename }}

\end{enumerate}


\clearpage


\section{Results}
\BLOCK{ for result_summary in result_summaries }
\input{\VAR{ result_summary }}

\BLOCK{ endfor }


\subsection{Summary results}
\VAR{ final_results }


\section{Conclusions}
\VAR{ conclusions_content }

\begin{thebibliography}{9}
\VAR{ bibliography_content }
\end{thebibliography}

\end{document}
